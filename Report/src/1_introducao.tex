\chapter[Introdução]{Introdução}
\label{cap:introducao}

A utilização de jogos para incentivar o interesse ao aprendizado em crianças é uma técnica que apresenta resultados bem positivos.~\footnote{Disponível em: \url{http://www.google.com}. Acesso em: 03 set. 2021.}

Com base nisso, pode-se utilizar recursos computacionais aliados a jogos de tabuleiro para apresentar utilizações práticas das tecnologias computacionais e despertar em crianças o interesse nas áreas de computação, elétrica e controle.

Além disso, estudos demonstram resultados positivos relacionados ao aprendizado de jogos como o Xadrez.

Correlacionando essas duas ideias, pode-se implementar um jogo de Xadrez que utiliza recursos tecnológicos para promover o aprendizado do jogo e incentivar a busca por conhecimento em áreas computacionais.

\section[Motivação]{Motivação}

Considerando a deficiência na educação básica no Brasil,~\footnote{Disponível em: \url{http://www.google.com}. Acesso em: 03 set. 2021.}, torna-se importante a busca por formas de incentivar o aprendizado e a busca por conhecimento por parte dos jovens.
Para isso, foi proposto o desenvolvimento de um sistema de controle de manipuladores robóticos para a utilização em jogos de Xadrez. Esse manipuladores podem, então, ser utilizados para apresentar o jogo e recursos tecnológicos para jovens brasileiros.

\section[Objetivos]{Objetivos}

Este trabalho tem como objetivo desenvolver um sistema de controle de manipuladores robóticos que permitam que dois jogadores participem em uma partida de Xadrez.

\section[Relevância]{Relevância}

Este trabalho é importante pois pode promover e instigar a busca por conhecimento em áreas de computação, elétrica e controle por parte de jovens brasileiros.