\chapter[Introdução]{Introdução}
\label{cap:introducao}

Atualmente, existe uma grande procura por funcionários especializados em Tecnologia da Informação (TI) e áreas similares, sendo percebida no mundo todo uma grande carência de profissionais qualificados para atuar nessas áreas.
Apesar do grande crescimento nas áreas de STEM (Science, Technology, Engineering and Mathematics) [Ciência, Tecnologia, Engenharia e Matemática], o número de profissionais qualificados nessas áreas não acompanha esse crescimento e a perspectiva é de que essa situação se acentue ainda mais no futuro \cite{shortage_of_workers}.

Devido a essa carência de profissionais, torna-se importante a busca por formas de incentivar o aprendizado e a busca por conhecimento por parte dos jovens.
Visando solucionar esse problema, foi decidido realizar um trabalho que incorpore conceitos de robótica, já que seu uso em atividades com crianças consegue influenciar positivamente o desenvolvimento de habilidades da área de STEM \cite{technology_for_stem}.

Com base nisso, foi proposto realizar o desenvolvimento de uma plataforma que utilize recursos computacionais passível de ser utilizada para demonstrar conceitos nas áreas de computação, elétrica e controle.
Para aumentar o interesse por esta plataforma foi definido que ela deve permitir que os participantes joguem uma partida de Xadrez.
Correlacionando essas ideias, implementou-se um jogo de Xadrez que pode ser jogado através de braços robóticos.

Este trabalho visa desenvolver um sistema de controle de manipuladores robóticos que permitam que dois jogadores participem em uma partida de Xadrez.
O sistema deve ser intuitivo e fácil de ser utilizado, de forma que os jogadores não precisem de conhecimentos avançados para utilizá-lo.
Além disso, os materiais utilizados devem ser de baixo custo, para facilitar a sua produção e demonstrar que é possível desenvolver um produto de qualidade utilizando materiais baratos.

Com o desenvolvimento dessa plataforma, será possível demonstrar conceitos de computação, elétrica e controle de forma prática e divertida.
Ela poderá ser facilmente transportada para diferentes locais e apresentada em eventos, como feiras de ciências e escolas.
Dessa forma, ela pode promover e instigar a busca por conhecimento, além de atrair futuros profissionais para a área de TI.