\chapter[Conclusões Preliminares]{Conclusões Preliminares}
\label{cap:conclusoes}

Durante o desenvolvimento do projeto, foi possível finalizar a placa de controle e um \textit{software} básico que permite o controle independente de cada junta.
Nessa primeira etapa de desenvolvimento, não foi possível implementar a lógica do jogo de Xadrez, tarefa que será realizada na próxima etapa.

O processo de produção da placa de controle foi mais demorado do que o inicialmente esperado, em parte pela necessidade de um cuidado especial para garantir que ambas as camadas estejam alinhadas.
Além disso, o processo de perfuração e soldagem da placa foi um pouco complicado e demorado.
Apesar disso, o resultado final foi satisfatório e a placa de controle apresenta um funcionamento adequado para o controle dos manipuladores.

O \textit{software} desenvolvido para o controle dos manipuladores, apesar de funcional, não facilita o controle deles.
Realizar o controle independente de cada junta é um processo complicado e pode provocar um desinteresse dos jogadores pelo produto desenvolvido.
Para a próxima etapa, esse \textit{software} será melhorado para que o controle dos manipuladores seja realizado apenas em duas dimensões de forma que o microcontrolador será responsável pelo cálculo dos ângulos das juntas.

A integração dos microcontroladores com o computador também será desenvolvida na próxima etapa do trabalho.
Essa integração será responsável por verificar se os movimentos realizados pelos jogadores são validos e por identificar quando um jogador ganhou ou perdeu o jogo.