\chapter[Introdução]{Introdução}
\label{cap:introducao}

Atualmente, existe uma grande procura por funcionários especializados em Tecnologia da Informação (TI) e áreas similares,
sendo percebida no mundo todo uma grande carência de profissionais qualificados para atuar nessas áreas, e a perspectiva é de que essa situação se acentue ainda mais no futuro \cite{shortage_of_workers}.

Com base nisso, foi proposto realizar o desenvolvimento de uma plataforma que utilize recursos computacionais passível de ser utilizada para demonstrar conceitos nas áreas de computação, elétrica e controle.
Para aumentar o interesse por ela foi definido que deve permitir que os participantes joguem uma partida de Xadrez.

Correlacionando essas ideias, foi decidido implementar um jogo de Xadrez que pode ser jogado através de braços robóticos.

\section[Motivação]{Motivação}

Considerando a carência de profissionais de TI no mercado, torna-se importante a busca por formas de incentivar o aprendizado e a busca por conhecimento por parte dos jovens.
Para tornar o aprendizado mais atrativo e divertido, foi feita a incorporação de um jogo no projeto proposto.
Finalmente, foi decidido que o projeto deveria usar elementos da robótica, visto que pesquisas demonstram que seu uso em atividades com crianças consegue influenciar positivamente o desenvolvimento de habilidades da área de STEM \cite{technology_for_stem}.

\section[Objetivos]{Objetivos}

Este trabalho visa desenvolver um sistema de controle de manipuladores robóticos que permitam que dois jogadores participem em uma partida de Xadrez.

Caso haja disponibilidade de tempo, o sistema também possibilitará que o jogo seja jogado por um jogador humano e um computador ou jogador humano através da Internet.

\section[Relevância]{Relevância}

Com o desenvolvimento dessa plataforma, será possível demonstrar conceitos de computação, elétrica e controle de forma prática e divertida.
Ela pode ser facilmente transportada para diferentes locais e apresentada em eventos, como feiras de ciências, por exemplo.
Dessa forma, ela pode promover e instigar a busca por conhecimento, além de atrair futuros profissionais para a área de TI.