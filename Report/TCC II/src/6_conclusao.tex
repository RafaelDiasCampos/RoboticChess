\chapter[Conclusão]{Conclusão}
\label{cap:conclusao}

A partir do desenvolvimento deste trabalho, foi possível obter uma plataforma simples e de baixo custo que permite o jogo entre duas pessoas ou entre uma pessoa e o computador.
Essa plataforma pode ser apresentada para crianças e jovens em feiras e eventos com o objetivo de introduzir conceitos básicos de computação, engenharia elétrica e controle de sistemas, além de instigar o conhecimento nessas áreas.

\section[Performance]{Performance}
\label{sec:performance}

Após finalizar o projeto, foi feita uma análise da performance do sistema,
com o objetivo de verificar quanto tempo o manipulador robótico leva para realizar um movimento.

Para isso, foi jogada uma partida contra o computador, anotando o tempo gasto para cada movimento realizado pelo manipulador robótico.
O resultados obtidos estão apresentados no Apêndice \ref{apendice:performance}.

\section[Limitações]{Limitações}
\label{sec:limitacoes}

O projeto apresenta algumas limitações que podem ser melhoradas em trabalhos futuros.

Em primeiro lugar, o manipulador robótico escolhido não apresenta tamanho suficiente para alcançar todas as casas do tabuleiro,
o que limita o jogo, principalmente quando uma pessoa está controlando o dispositivo.
A escolha de um modelo de manipulador robótico com maior alcance pode resolver esse problema.

Além disso, o projeto não apresenta um sistema de detecção de peças no tabuleiro,
o que faz com que o usuário tenha que informar ao computador qual peça foi movida.
A implementação de um sistema de visão computacional pode tornar o sistema muito mais interativo e intuitivo.