%% Baseado no arquivo: 
%% abtex2-modelo-trabalho-academico.tex, v-1.9.6 laurocesar
%% by abnTeX2 group at http://www.abntex.net.br/ 
%% Adaptado para um modelo de TCC (Graduação)

% ---
% Capa
% ---
\imprimircapa
% ---

% ---
% Folha de rosto
% (o * indica que haverá a ficha bibliográfica)
% ---
\imprimirfolhaderosto*
% ---

% ---
% Inserir a ficha bibliografica
% ---

% Isto é um exemplo de Ficha Catalográfica, ou ``Dados internacionais de
% catalogação-na-publicação''. Você pode utilizar este modelo como referência. 
% Porém, provavelmente a biblioteca da sua universidade lhe fornecerá um PDF
% com a ficha catalográfica definitiva após a defesa do trabalho. Quando estiver
% com o documento, salve-o como PDF no diretório do seu projeto e substitua todo
% o conteúdo de implementação deste arquivo pelo comando abaixo:
%
% \begin{fichacatalografica}
%     
% \end{fichacatalografica}

\begin{fichacatalografica}
~
%\includepdf{fig_ficha_catalografica.pdf}
% 	\sffamily
% 	\vspace*{\fill}					% Posição vertical
% 	\begin{center}					% Minipage Centralizado
% 	\fbox{\begin{minipage}[c][8cm]{13.5cm}		% Largura
% 	\small
% 	\imprimirautor
% 	%Sobrenome, Nome do autor
	
% 	\hspace{0.5cm} \imprimirtitulo  / \imprimirautor. --
% 	\imprimirlocal, \imprimirdata-
	
% 	\hspace{0.5cm} \pageref{LastPage} p. : il. (algumas color.) ; 30 cm.\\
	
% 	\hspace{0.5cm} \imprimirorientadorRotulo~\imprimirorientador\\
	
% 	\hspace{0.5cm}
% 	\parbox[t]{\textwidth}{\imprimirtipotrabalho~--~\imprimirinstituicao,
% 	\imprimirdata.}\\
	
% 	\hspace{0.5cm}
% 		1. Palavra-chave1.
% 		2. Palavra-chave2.
% 		2. Palavra-chave3.
% 		I. Orientador.
% 		II. Universidade xxx.
% 		III. Faculdade de xxx.
% 		IV. Título 			
% 	\end{minipage}}
% 	\end{center}
\end{fichacatalografica}
% ---

% % ---
% % Inserir errata
% % ---
% \begin{errata}
% Elemento opcional da \citeonline[4.2.1.2]{NBR14724:2011}. Exemplo:

% \vspace{\onelineskip}

% FERRIGNO, C. R. A. \textbf{Tratamento de neoplasias ósseas apendiculares com
% reimplantação de enxerto ósseo autólogo autoclavado associado ao plasma
% rico em plaquetas}: estudo crítico na cirurgia de preservação de membro em
% cães. 2011. 128 f. Tese (Livre-Docência) - Faculdade de Medicina Veterinária e
% Zootecnia, Universidade de São Paulo, São Paulo, 2011.

% \begin{table}[htb]
% \center
% \footnotesize
% \begin{tabular}{|p{1.4cm}|p{1cm}|p{3cm}|p{3cm}|}
%   \hline
%    \textbf{Folha} & \textbf{Linha}  & \textbf{Onde se lê}  & \textbf{Leia-se}  \\
%     \hline
%     1 & 10 & auto-conclavo & autoconclavo\\
%    \hline
% \end{tabular}
% \end{table}

% \end{errata}
% ---

% ---
% inserir lista de abreviaturas e siglas
% ---
\begin{siglas}
  \item[CEFET-MG] Centro Federal de Educação Tecnológica de Minas Gerais
  \item[STEM] \textit{Science, Technology, Engineering and Mathematics} [Ciência, Tecnologia, Engenharia e Matemática]
\end{siglas}
% ---